\documentclass[final]{ols}

% These two packages allow easy handling of urls and identifiers per the example paper.
\usepackage{url}
\usepackage{zrl}

% The following package is not required, but is a handy way to put PDF and EPS graphics
% into your paper using the \includegraphics command.
\ifpdf
\usepackage[pdftex]{graphicx}
\else
\usepackage{graphicx}
\fi


\begin{document}

% Mandatory: article title specification.
% Do not put line breaks or other clever formatting in \title or
% \shortauthor; these are moving arguments.

\title{Linux Kernel Development}
\subtitle{How Fast it is Going, Who is Doing It, What They are Doing, and Who is Sponsoring It}
%\subtitle{ }  % Subtitle is optional.
\date{}             % You can put a fixed date in if you wish,
                    % allow LaTeX to use the date of typesetting,
                    % or use \date{} to have no date at all.
                    % Whatever you do, there will not be a date
                    % shown in the proceedings.

\shortauthor{Greg  Kroah-Hartman}  % Just you and your coauthors' names.
% for example, \shortauthor{A.N.\ Author and A.\ Nother}
% or perchance \shortauthor{Smith, Jones, Black, White, Gray, \& Greene}

\author{%  Authors, affiliations, and email addresses go here, like this:
Greg Kroah-Hartman \\
{\itshape SuSE Labs / Novell Inc.}\\
{\ttfamily\normalsize gregkh@suse.de}\\
% \and
% Bob \\
% {\itshape Bob's affiliation.}\\
% {\ttfamily\normalsize bob@example.com}\\
} % end author section

\maketitle

%\begin{abstract}
% Article abstract goes here.
%\input{kroah-hartman-abstract.tex}
% This paper will analyze the changes that have been
% happening in the Linux kernel since the 2.6.0 release.
% It will detail the following things:
% 
% - the rate of change over time
% 
% - where the changes are occurring
% 
% - who is doing the changes
% 
% - what companies are doing the changes
% 
% - what companies and people are doing what changes
% 
% It will also describe the flow of patches between
% developers by analyzing the "Signed-off-by:" patch, and
% create a very large diagram showing all of the
% different individuals contributing to the Linux kernel
% over the past few years, and how they are
% interconnected to each other.
% 
% Lots of nice graphs and charts and even posters will be
% created for this paper and presentation.
% 
% All of the data generated by this paper and
% presentation will be made available for anyone else to
% use it to analyze the results in whatever way they
% might wish.
% 
%\end{abstract}

% Body of your article goes here.  You are mostly unrestricted in what
% LaTeX features you can use; however, the following will not work:
% \thispagestyle
% \marginpar
% table of contents
% list of figures / tables
% glossaries
% indices

\section{Introduction}
The Linux kernel is one of the most popular Open Source development
projects, and yet not much attention has been placed on who is doing
this development, who is sponsoring this development, and what exactly
is being developed.  This paper should help explain some of these facts
by delving into the kernel changelogs and producing lots of statistics.

This paper will focus on the kernel releases of the past two and 1/3
years, from the 2.6.11 through the 2.6.21 release.

\section{Development vs. Stability}
In the past, the Linux kernel was split into two different trees, the
\textit{development} branch, and the \textit{stable} branch.  The development branch
was specified by using an odd number for the second release number, while
the stable branch used an even number.  As an example, the 2.5.32
release was a development release, while the 2.4.24 release is a stable
release.

After the 2.6 kernel series was created, the developers decided to
change this method of having two different trees.  They declared that
all 2.6 kernel releases would be considered ``stable,'' no matter how
quickly development was happening.  These releases would happen every 2
to 3 months and would allow developers to add new features and then
stabilize them in time for the next release.  This was done in order to
allow distributions to be able to decide on a release point easier by
always having at least one stable kernel release near a distribution
release date.

To help with stability issues while the developers are creating a new
kernel version, a \texttt{-stable} branch was created that would contain bug
fixes and security updates for the past kernel release before the next
major release happened.

This is best explained with the diagram shown in Figure~\ref{gkh-release-cycle}.
The kernel team
released the 2.6.19 kernel as a stable release.  Then the developers
started working on new features and started releasing the \texttt{-rc} versions
as development kernels so that people could help test and debug the
changes.  After everyone agreed that the development release was stable
enough, it was released as the 2.6.20 kernel.


\begin{figure}[thb]
\begin{center}
\includegraphics[width=0.9\columnwidth]{fig-release_cycle}
\caption{Kernel release cycles}
\label{gkh-release-cycle}
\end{center}
\end{figure}

While the development of new features was happening, the 2.6.19.1,
2.6.19.2, and other stable kernel versions were released, containing bug
fixes and security updates.

For this paper, we are going to focus on the main kernel releases, and
ignore the \texttt{-stable} releases, as they contain a very small number of
bugfixes and are not where any development happens.

\section{Frequency of release}
When the kernel developers first decided on this new development cycle,
it was said that a new kernel would be released every 2-3 months, in
order to prevent lots of new development from being ``backed up.''
The actual number of days between releases can be seen in Table~\ref{gkh-days}.

\begin{table}[tbph]
\begin{center}
\begin{tabular}{|r|r|}
\hline
\multicolumn{1}{|c|}{Kernel} & \multicolumn{1}{|c|}{Days of}\\
\multicolumn{1}{|c|}{Version} & \multicolumn{1}{|c|}{development}\\
\hline
\hline
2.6.11	& 69	\\
2.6.12	& 108	\\
2.6.13	& 73	\\
2.6.14	& 61	\\
2.6.15	& 68	\\
2.6.16	& 77	\\
2.6.17	& 91	\\
2.6.18	& 95	\\
2.6.19	& 72	\\
2.6.20	& 68	\\
2.6.21	& 81	\\
\hline
\end{tabular}
\caption{Frequency of kernel releases}
\label{gkh-days}
\end{center}
\end{table}


It turns out that they were very correct, with the average being 2.6
months between releases.

\section{Rate of Change}

When modifying the Linux kernel, developers break their changes down
into small, individual units of change, called patches.  These patches
usually do only one thing to the source tree, and are built on top of
each other, modifying the source code by changing, adding, or removing
lines of code.  At each change point in time, the kernel should be able
to be successfully built and operate.  By enforcing this kind of
discipline, the kernel developers must break their changes down into
small logical pieces.  The number of individual changes that go into
each kernel release is very large, as can be seen in Table~\ref{gkh-changes}.

\begin{table}%[tbph]
\begin{center}
\begin{tabular}{|r|c|}
\hline
\multicolumn{1}{|c|}{Kernel}  & \multicolumn{1}{|c|}{Changes per} \\
 \multicolumn{1}{|c|}{Version} &  \multicolumn{1}{|c|}{Release}\\
\hline
\hline
\rule[-0.1ex]{0pt}{2.5ex}2.6.11	& 4,041	\\
2.6.12	& 5,565	\\
2.6.13	& 4,174	\\
2.6.14	& 3,931	\\
2.6.15	& 5,410	\\
2.6.16	& 5,734	\\
2.6.17	& 6,113	\\
2.6.18	& 6,791	\\
2.6.19	& 7,073	\\
2.6.20	& 4,983	\\
2.6.21	& 5,349	\\
\hline
\end{tabular}
\caption{Changes per kernel release}
\label{gkh-changes}
\end{center}
\end{table}

When you compare the number of changes per release, with the length of
time for each release, you can determine the number of changes per hour, as can
be seen in Table~\ref{gkh-hour}.

\begin{table}%[hbtp]
\begin{center}
\begin{tabular}{|c|c|}
\hline
\multicolumn{1}{|c|}{Kernel}  & \multicolumn{1}{|c|}{Changes} \\
\multicolumn{1}{|c|}{Version} & \multicolumn{1}{|c|}{per Hour}\\
\hline
\hline
\rule[-0.1ex]{0pt}{2.5ex}2.6.11	& 2.44 	\\
2.6.12	& 2.15 	\\
2.6.13	& 2.38 	\\
2.6.14	& 2.69 	\\
2.6.15	& 3.31 	\\
2.6.16	& 3.10 	\\
2.6.17	& 2.80 	\\
2.6.18	& 2.98 	\\
2.6.19	& 4.09 	\\
2.6.20	& 3.05 	\\
2.6.21	& 2.75 	\\
\hline
\end{tabular}
\caption{Changes per hour by kernel release}
\label{gkh-hour}
\end{center}
\end{table}

So, from the 2.6.11 to the 2.6.21 kernel release, a total of 852 days,
there were 2.89 patches applied to the kernel tree per hour.  And that
is only the patches that were accepted.


\section {Kernel Source Size}

The Linux kernel keeps growing in size over time, as more hardware is
supported, and new features added.  For the following numbers, I count
everything in the released Linux source tarball as ``source code'' even
though a small percentage is the scripts used to configure and build the
kernel, as well as a minor amount of documentation.  This is done
because someone creates those files, and are worthy of being mentioned.

The information in Table~\ref{gkh-lines} show the number of files and
lines in each kernel version.

\begin{table}[tbph]
\begin{center}
\setlength{\tabcolsep}{2.0ex}
\begin{tabular}{|r|r|r|}
\hline
\multicolumn{1}{|p{1.3cm}|}{\centering{Kernel Version}} & 
\multicolumn{1}{|c|}{\raisebox{-1.5ex}{Files}} & 
\multicolumn{1}{|c|}{\raisebox{-1.5ex}{Lines}} \\
%\multicolumn{1}{|c|}{Version} & ~ & ~ \\
\hline
\hline
\rule[-0.1ex]{0pt}{2.5ex}2.6.11	& 17,091	& 6,624,076	\\
2.6.12	& 17,361	& 6,777,860	\\
2.6.13	& 18,091	& 6,988,800	\\
2.6.14	& 18,435	& 7,143,233	\\
2.6.15	& 18,812	& 7,290,070	\\
2.6.16	& 19,252	& 7,480,062	\\
2.6.17	& 19,554	& 7,588,014	\\
2.6.18	& 20,209	& 7,752,846	\\
2.6.19	& 20,937	& 7,976,221	\\
2.6.20	& 21,281	& 8,102,533	\\
2.6.21	& 21,615	& 8,246,517	\\
\hline
\end{tabular}
\caption{Size per kernel release}
\label{gkh-lines}
\end{center}
\end{table}

Over these releases, the kernel team has a very constant growth rate of
about 10\% per year, a very impressive number given the size of the code
tree.

When you combine the number of lines added per release, and compare it
to the amount of time per release, you can get some very impressive
numbers, as can be seen in Table~\ref{gkh-lines-per-hour}.

\begin{table}[tbph]
\begin{center}
\begin{tabular}{|r|r|}
\hline
\multicolumn{1}{|c|}{Kernel} & \multicolumn{1}{|c|}{Lines per}\\
\multicolumn{1}{|c|}{Version} & \multicolumn{1}{|c|}{Hour}\\
\hline
\hline
\rule[-0.1ex]{0pt}{2.5ex}2.6.11	& 77.6\hspace*{1ex}		\\
2.6.12	& 59.3\hspace*{1ex}		\\
2.6.13	& 120.4\hspace*{1ex}		\\
2.6.14	& 105.5\hspace*{1ex}		\\
2.6.15	& 90.0\hspace*{1ex}		\\
2.6.16	& 102.8\hspace*{1ex}		\\
2.6.17	& 49.4\hspace*{1ex}		\\
2.6.18	& 72.3\hspace*{1ex}		\\
2.6.19	& 129.3\hspace*{1ex}		\\
2.6.20	& 77.4\hspace*{1ex}		\\
2.6.21	& 74.1\hspace*{1ex}		\\
\hline
\end{tabular}
\caption{Lines per hour by kernel release}
\label{gkh-lines-per-hour}
\end{center}
\end{table}


Summing up these numbers, it comes to a crazy 85.63 new lines of code
being added to the kernel tree every hour for the past 2 1/3 years.

\section{Where the Change is Happening}

The Linux kernel source tree is highly modular, enabling new drivers and
new architectures to be added quite easily.  The source code can be
broken down into the following categories:

\begin {itemize}
\item \textbf{core}: this is the core kernel code, run by everyone and
included in all architectures.  This code is located in the
subdirectories
\texttt{block/},
\texttt{ipc/},
\texttt{init/},
\texttt{kernel/},
\texttt{lib/},
\texttt{mm/},
and portions of the
\texttt{include/} directory.

\item \textbf{drivers}: these are the drivers for different hardware and
virtual devices.  This code is located in the subdirectories
\texttt{crypto/},
\texttt{drivers/},
\texttt{sound/},
\texttt{security/},
and portions of the
\texttt{include/} directory.

\item \textbf{architecture}: this is the CPU specific code, where
anything that is only for a specific processor lives.  This code is
located in the
\texttt{arch/},
and portions of the
\texttt{include/} directory.

\item \textbf{network}: this is the code that controls the different
networking protocols.  It is located in the
\texttt{net/}
directory and the
\texttt{include/net} subdirectory.

\item \textbf{filesystems}: this is the code that controls the different
filesystems.  It is located in the
\texttt{fs/} directory.

\item \textbf{miscellaneous}: this is the rest of the kernel source
code, including the code needed to build the kernel, and the
documentation for various things.  It is located in
\texttt{Documentation/},
\texttt{scripts/},
and
\texttt{usr/} directories.

\end {itemize}

The breakdown of the 2.6.21 kernel's source tree by the number of
different files in the different category is shown in
Table~\ref{gkh-size-by-files}, while Table~\ref{gkh-size-by-lines} shows
the breakdown by the number of lines of code.

\begin{table}[tbph]
\begin{center}
\begin{tabular}{|l|r|r|}
\hline
\raisebox{-1.5ex}{Category} & 
\multicolumn{1}{|c|}{\raisebox{-1.5ex}{Files}} & 
\multicolumn{1}{|p{1.2cm}|}{\centering{\% of kernel}}  \\
\hline
\hline
\rule[-0.1ex]{0pt}{2.5ex}core		&  1,371	&	 6\%	\\
drivers		&  6,537	&	30\%	\\
architecture	& 10,235	&	47\%	\\
network		&  1,095	&	 5\%	\\
filesystems	&  1,299	&	 6\%	\\
miscellaneous	&  1,068	&	 5\%	\\
\hline
\end{tabular}
\caption{2.6.21 Kernel size by files}
\label{gkh-size-by-files}
\end{center}
\end{table}

\begin{table}[tbph]
\begin{center}
\begin{tabular}{|l|r|r|}
\hline
\raisebox{-1.5ex}{Category} & 
\multicolumn{1}{|p{2.0cm}|}{\centering{Lines of Code}} &
\multicolumn{1}{|p{1.2cm}|}{\centering{\% of kernel}} \\
%~  & \multicolumn{1}{|c|}{of Code} & \multicolumn{1}{|c|}{kernel} \\
\hline
\hline
\rule[-0.1ex]{0pt}{2.5ex}core		&   330,637 &  4\%	\\
drivers		& 4,304,859 & 52\%	\\
architecture	& 2,127,154 & 26\%	\\
network		&   506,966 &  6\%	\\
filesystems	&   702,913 &  9\%	\\
miscellaneous	&   263,848 &  3\%	\\
\hline
\end{tabular}
\caption{2.6.21 Kernel size by lines of code}
\label{gkh-size-by-lines}
\end{center}
\end{table}

In the 2.6.21 kernel release, the architecture section of the kernel
contains the majority of the different files, but the majority of the
different lines of code are by far in the drivers section.

I tried to categorize what portions of the kernel are changing over
time, but there did not seem to be a simple way to represent the
different sections changing based on kernel versions.  Overall, the
percentage of change seemed to be evenly spread based on the percentage
that the category took up within the overall kernel structure.

\section{Who is Doing the Work}

The number of different developers who are doing Linux kernel
development, and the identifiable companies\footnote{The identification
of the different companies is described in the next section.} who are
sponsering this work, has been slowly increasing over the different
kernel versions, as can be seen in Table~\ref{gkh-num-developers}.

\begin{table}%[tbph]
\begin{center}
\begin{tabular}{|r|r|r|}
\hline
\multicolumn{1}{|c|}{Kernel}  & \multicolumn{1}{|c|}{Number of}  & \multicolumn{1}{|c|}{Number of} \\
\multicolumn{1}{|c|}{Version} & \multicolumn{1}{|c|}{Developers} & \multicolumn{1}{|c|}{Companies}\\
\hline
\hline
\rule[-0.1ex]{0pt}{2.5ex}2.6.11	& 479\hspace{2ex}	& 30\hspace{3ex}	\\
2.6.12	& 704\hspace{2ex}	& 38\hspace{3ex}	\\
2.6.13	& 641\hspace{2ex}	& 39\hspace{3ex}	\\
2.6.14	& 632\hspace{2ex}	& 45\hspace{3ex}	\\
2.6.15	& 685\hspace{2ex}	& 49\hspace{3ex}	\\
2.6.16	& 782\hspace{2ex}	& 56\hspace{3ex}	\\
2.6.17	& 787\hspace{2ex}	& 54\hspace{3ex}	\\
2.6.18	& 904\hspace{2ex}	& 60\hspace{3ex}	\\
2.6.19	& 887\hspace{2ex}	& 67\hspace{3ex}	\\
2.6.20	& 730\hspace{2ex}	& 75\hspace{3ex}	\\
2.6.21	& 838\hspace{2ex}	& 68\hspace{3ex}	\\
\hline
\rule[-0.1ex]{0pt}{2.5ex}All	& 2998\hspace{2ex}	& 83\hspace{3ex}	\\
\hline
\end{tabular}
\caption{Number of individual developers and employers}
\label{gkh-num-developers}
\end{center}
\end{table}

Factoring in the amount of time between each individual kernel releases
and the number of developers and employers ends up showing that there
really is an increase of the size of the community, as can be shown in
Table~\ref{gkh-num-developers-over-time}.

\begin{table}%[tbph]
\begin{center}
\begin{tabular}{|r|r|r|}
\hline
\multicolumn{1}{|p{1.3cm}|}{\centering{\vspace*{0.3ex}Kernel Version}}  & 
\multicolumn{1}{|p{2.0cm}|}{\centering{Number of Developers per day}}  & 
\multicolumn{1}{|p{2.0cm}|}{\centering{Number of Companies per day}} \\
%\multicolumn{1}{|c|}{Kernel}  & \multicolumn{1}{|c|}{Number of}  & \multicolumn{1}{|c|}{Number of} \\
%\multicolumn{1}{|c|}{Version} & \multicolumn{1}{|c|}{Developers} & \multicolumn{1}{|c|}{Companies}\\
%\multicolumn{1}{|c|}{}        & \multicolumn{1}{|c|}{per day}    & \multicolumn{1}{|c|}{per day}\\
\hline
\hline
\rule[-0.1ex]{0pt}{2.5ex}2.6.11	&  6.94\hspace{3ex}	& 0.43\hspace{3ex}	\\
2.6.12	&  6.52\hspace{3ex}	& 0.35\hspace{3ex}	\\
2.6.13	&  8.78\hspace{3ex}	& 0.53\hspace{3ex}	\\
2.6.14	& 10.36\hspace{3ex}	& 0.74\hspace{3ex}	\\
2.6.15	& 10.07\hspace{3ex}	& 0.72\hspace{3ex}	\\
2.6.16	& 10.16\hspace{3ex}	& 0.73\hspace{3ex}	\\
2.6.17	&  8.65\hspace{3ex}	& 0.59\hspace{3ex}	\\
2.6.18	&  9.52\hspace{3ex}	& 0.63\hspace{3ex}	\\
2.6.19	& 12.32\hspace{3ex}	& 0.93\hspace{3ex}	\\
2.6.20	& 10.74\hspace{3ex}	& 1.10\hspace{3ex}	\\
2.6.21	& 10.35\hspace{3ex}	& 0.84\hspace{3ex}	\\
\hline
\end{tabular}
\caption{Number of individual developers and employers over time}
\label{gkh-num-developers-over-time}
\end{center}
\end{table}

Despite this large number of individual developers, there is still a
small number who are doing the majority of the work.  Over the past two
and one half years, the top 10 individual developers have contributed
15 percent of the number of changes and the top 30 developers have
contributed 30 percent.  The list of individual developers, the number
of changes they have contributed, and the percentage of the overall
total can be seen in Table~\ref{gkh-individuals}.

\begin{table}[btph]
\begin{center}
\begin{small}
\newlength{\jwlNL}
\setlength{\jwlNL}{0.4\columnwidth}
\newlength{\jwlCL}
\setlength{\jwlCL}{0.18\columnwidth}
\begin{tabular}{|r|r|r|}
\hline
\multicolumn{1}{|p{\jwlNL}|}{\raisebox{-3ex}{Name}} & 
\multicolumn{1}{|p{\jwlCL}|}{\centering{Number of Changes}}  & 
\multicolumn{1}{|p{\jwlCL}|}{\centering{\vspace{-0.3ex}Percent of Total}} \\
%\multicolumn{1}{|c|}{}     & \multicolumn{1}{|c|}{Changes}    & \multicolumn{1}{|c|}{Total} \\
\hline
\hline
\rule[-0.1ex]{0pt}{2.5ex}Al Viro                  & 1326 & 2.2\% \\
David S. Miller          & 1096 & 1.9\% \\
Adrian Bunk              & 1091 & 1.8\% \\
Andrew Morton            &  991 & 1.7\% \\
Ralf Baechle             &  981 & 1.7\% \\
Andi Kleen               &  856 & 1.4\% \\
Russell King             &  788 & 1.3\% \\
Takashi Iwai             &  764 & 1.3\% \\
Stephen Hemminger        &  650 & 1.1\% \\
Neil Brown               &  626 & 1.1\% \\
\hline
\rule[-0.1ex]{0pt}{2.5ex}Tejun Heo                &  606 & 1.0\% \\
Patrick McHardy          &  529 & 0.9\% \\
Randy Dunlap             &  486 & 0.8\% \\
Jaroslav Kysela          &  463 & 0.8\% \\
Trond Myklebust          &  445 & 0.8\% \\
Jean Delvare             &  436 & 0.7\% \\
Christoph Hellwig        &  435 & 0.7\% \\
Linus Torvalds           &  433 & 0.7\% \\
Ingo Molnar              &  429 & 0.7\% \\
Jeff Garzik              &  424 & 0.7\% \\
\hline
\rule[-0.1ex]{0pt}{2.5ex}David Woodhouse          &  413 & 0.7\% \\
Paul Mackerras           &  411 & 0.7\% \\
David Brownell           &  398 & 0.7\% \\
Jeff Dike                &  397 & 0.7\% \\
Ben Dooks                &  392 & 0.7\% \\
Greg Kroah-Hartman       &  388 & 0.7\% \\
Herbert Xu               &  376 & 0.6\% \\
Dave Jones               &  371 & 0.6\% \\
Ben Herrenschmidt        &  365 & 0.6\% \\
Mauro  Chehab    &  365 & 0.6\% \\
\hline
\end{tabular}
\end{small}
\caption{Individual Kernel contributors}
\label{gkh-individuals}
\end{center}
\end{table}


\section{Who is Sponsoring the Work}

Despite the broad use of the Linux kernel in a wide range of different
types of devices, and reliance of it by a number of different companies,
the number of individual companies that help sponsor the development of
the Linux kernel remains quite small as can be seen by the list of
different companies for each kernel version in Table~\ref{gkh-num-developers}.

The identification of the different companies was deduced through the
use of company email addresses and the known sponsoring of some
developers.  It is possible that a small number of different companies
were missed, however based on the analysis of the top contributors of
the kernel, the majority of the contributions are attributed in this
paper.

The large majority of contributions still come from individual
contributors, either because they are students, they are contributing on
their own time, or their employers are not allowing them to use their
company email addresses for their kernel development efforts.  As seen
in Table~\ref{gkh-num-companies} almost half of the contributions are
done by these individuals.

\begin{table}[h!]%[tbph]
\begin{center}
\begin{small}
\begin{tabular}{|r|r|r|}
\hline
\multicolumn{1}{|c|}{Company} & \multicolumn{1}{|c|}{Number of}  & \multicolumn{1}{|c|}{Percent of} \\
\multicolumn{1}{|c|}{Name}    & \multicolumn{1}{|c|}{Changes}    & \multicolumn{1}{|c|}{Total} \\
\hline
\hline
\rule[-0.1ex]{0pt}{2.5ex}Unknown                  & 27976 & 47.3\% \\
Red Hat                  &  6106 & 10.3\% \\
Novell                   &  5923 & 10.0\% \\
Linux Foundation         &  4843 & 8.2\%  \\
IBM                      &  3991 & 6.7\%  \\
Intel                    &  2244 & 3.8\%  \\
SGI                      &  1353 & 2.3\%  \\
NetApp                   &   636 & 1.1\%  \\
Freescale                &   454 & 0.8\%  \\
linutronix               &   370 & 0.6\%  \\
\hline
\rule[-0.1ex]{0pt}{2.5ex}HP                         & 360 & 0.6\% \\
Harvard                    & 345 & 0.6\% \\
SteelEye                   & 333 & 0.6\% \\
Oracle                     & 319 & 0.5\% \\
Conectiva                  & 296 & 0.5\% \\
MontaVista                 & 291 & 0.5\% \\
Broadcom                   & 285 & 0.5\% \\
Fujitsu                    & 266 & 0.4\% \\
Veritas                    & 219 & 0.4\% \\
QLogic                     & 218 & 0.4\% \\
\hline
\rule[-0.1ex]{0pt}{2.5ex}Snapgear                   & 214 & 0.4\% \\
Emulex                     & 147 & 0.2\% \\
LSI Logic                  & 130 & 0.2\% \\
SANPeople                  & 124 & 0.2\% \\
Qumranet                   & 106 & 0.2\% \\
Atmel                      &  91 & 0.2\% \\
Toshiba                    &  90 & 0.2\% \\
Samsung                    &  82 & 0.1\% \\
Renesas Technology         &  81 & 0.1\% \\
VMWare                     &  78 & 0.1\% \\
\hline
\end{tabular}
\end{small}
\caption{Company Kernel Contributions}
\label{gkh-num-companies}
\end{center}
\end{table}



\section{Conclusion}

The Linux kernel is one of the largest and most successful open source
projects that has ever come about.  The huge rate of change and number
of individual contributors show that it has a vibrant and active
community, constantly causing the evolution of the kernel to survive the
number of different environments it is used in.  However, despite the
large number of individual contributors, the sponsorship of these
developers seem to be consolidated in a small number of individual
companies.  It will be interesting to see if, over time, the companies
that rely on the success of the Linux kernel will start to sponsor the
direct development of the project, to help ensure that it remains
valuable to those companies.

\section{Thanks}

The author would like to thank the thousands of individual kernel
contributors, without them, papers like this would not be interesting to
anyone.

I would also like to thank Jonathan Corbet, whose \texttt{gitdm} tool
were used to create a large number of these different statistics.
Without his help, this paper would have taken even longer to write, and
not been as informative.

\section{Resources}

The information for this paper was retrieved directly from the Linux
kernel releases as found at the \url{kernel.org} web site and from the
git kernel repository.  Some of the logs from the git repository were
cleaned up by hand due to email addresses changing over time, and minor
typos in authorship information.  A spreadsheet was used to compute a
number of the statistics.  All of the logs, scripts, and spreadsheet can
be found at
\url{http://www.kernel.org/pub/linux/kernel/people/gregkh/kernel_history/}


\end{document}

